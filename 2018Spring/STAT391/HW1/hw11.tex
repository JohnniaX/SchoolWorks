\documentclass[preprint,12pt]{elsarticle}

    \usepackage[sc]{mathpazo} % Use the Palatino font
    \usepackage[T1]{fontenc} % Use 8-bit encoding that has 256 glyphs
    \usepackage{microtype} % Slightly tweak font spacing for aesthetics
    \usepackage[english]{babel} % Language hyphenation and typographical rules
    \usepackage{booktabs} % Horizontal rules in tables
    \usepackage{enumitem} % Customized lists
    \usepackage[table,xcdraw]{xcolor}
    \usepackage[utf8]{inputenc} % Required for inputting international characters
    \usepackage{parskip}
    \usepackage{graphicx}
    \usepackage{hyperref}
    \usepackage{pdfpages}
    \usepackage{amsmath}
    \usepackage{esvect}
    \usepackage{listings}
    \usepackage{color}
    \usepackage[title]{appendix}
    \hypersetup{
        colorlinks=true,
        linkcolor=blue,
        filecolor=magenta,      
        urlcolor=cyan,
    }
    \definecolor{dkgreen}{rgb}{0,0.6,0}
    \definecolor{gray}{rgb}{0.5,0.5,0.5}
    \definecolor{mauve}{rgb}{0.58,0,0.82}
    \definecolor{lightgray}{rgb}{0.83, 0.83, 0.83}
    \definecolor{timberwolf}{rgb}{0.86, 0.84, 0.82}
    \definecolor{whitesmoke}{rgb}{0.96, 0.96, 0.96}
    
    \lstset{frame=tb,
    language=python,
    aboveskip=3mm,
    belowskip=3mm,
    showstringspaces=false,
    columns=flexible,
    basicstyle={\small\ttfamily},
    numbers=none,
    numberstyle=\tiny\color{gray},
    keywordstyle=\color{blue},
    commentstyle=\color{dkgreen},
    stringstyle=\color{mauve},
    breaklines=true,
    breakatwhitespace=true,
    tabsize=3,
    backgroundcolor = \color{whitesmoke}
    }

    \begin{document}
    \title{\LARGE \bf
        STAT 391 Homework 1
        }
        
        \author{ \parbox{3 in}{\centering Chongyi Xu \\
                 University of Washington\\
                 STAT 391 Spring 2018\\
                 {\tt\small chongyix@uw.edu}}
        }
    \maketitle

    \section{Problem 1- Practice with Probability}
    \begin{enumerate}[label=\alph*]
        \item Estimate $\theta = (\theta_0\ \theta_1\ \dots\ \theta_4)$
        \begin{lstlisting}
observations = 0
counter = {0:0, 1:0, 2:0, 3:0, 4:0}
for line in open(r'C:\Users\johnn\Documents\UW\SchoolWorks
\\2018Spring\STAT391\HW1\hw2-little-amazon.dat').readlines():
    line = int(line.rstrip())
    counter[line] = counter[line] + 1
    observations = observations + 1

theta = [counter[0]/observations, counter[1]/observations, \
            counter[2]/observations, counter[3]/observations, counter[4]/observations]
print(theta)
        \end{lstlisting}
        As the result, I got my $\theta$ to be
        \begin{equation*}
            \theta = (0.149, 0.396, 0.049, 0.255, 0.151)
        \end{equation*}
        And the sufficient statistics are the counts for each title, which is Table\ref{table1}
        \begin{table}[]
            \centering
            \caption{Sufficient Statistics for Books}
            \label{table1}
            \begin{tabular}{lll}
            Book ID & Book Title                          & Count \\
            0       & War and Peace                       & 149   \\
            1       & Harry Potter \& the Deathly Hallows & 396   \\
            2       & Winnie the Pooh                     & 49    \\
            3       & Get rich NOW                        & 255   \\
            4       & Probability                         & 151  
            \end{tabular}
        \end{table}

    \item A customer buys 3 books. What is the probability that he buys “War and Peace”, “Harry Potter”, 
    “Probability” in this order?
    Assign the event that a customer buys the $i^{th}$ book as $E_i$, then we are looking for the probability
    that $P(E_0) \cdot P(E_1) \cdot P(E_4)$ since the book that every time that customer gets is an independent random
    \begin{equation*}
        P(E_0) \cdot P(E_1) \cdot P(E_4) = 0.149 * 0.396 * 0.151 \approx 0.008910
    \end{equation*}
    And getting these three books has $2 * 3 = 6$ combinations, and we are only looking for one of those, thus
    \begin{equation*}
        P = \binom{6}{1} \cdot (P(E_0) \cdot P(E_1) \cdot P(E_4)) \approx 0.001485
    \end{equation*}
    Therefore, the probability that the customer buys “War and Peace”, “Harry Potter”, “Probability” in this order
    is $0.001484934$.

    \item A customer buys 4 books. What is the probability that she buys only non-fiction, that is, $N={3, 4}$
    ?
    Denote the event that the customer buys 4 books and she buys only non-fiction as $E$. Then
    \begin{equation*}
        P(E) = (P(E_3) + P(E_4))^4 = (0.255 + 0.151)^4 \approx 0.02717
    \end{equation*}

    \item A customer buys 2 "Probability" books and 3 fiction (i.e 0 or 1 or 2) books. What is the probability
    of this event?
    Denote the event that the customer buys 2 "Probability" books and 3 fiction (i.e 0 or 1 or 2) books as $E$. Then
    \begin{equation*}
        P(E) = P(E_4)^2 * (P(E_0) + P(E_1) + P(E_2))^3 = 0.151^2 * (0.149 + 0.396 + 0.049)^3 \approx 0.004779
    \end{equation*}

    \item A customer buys n books. What is the probability that he buys at least one "Probability"?
    Denote the event that the customer buys at least one "Probability" among n books he bought
    \begin{equation*}
        P(E) = 1 - P(\neg E) = 1 - (1 - 0.151)^n
    \end{equation*}
    \end{enumerate} 

    \section{Problem 2 Practice with probability}
    A man claims to have extrasensory perception. As a test, a fair coin is fipped 10 times, and the man is
    asked to predict the outcome in advance. He gets 7 out of 10 correct.
    \begin{table}[]
        \centering
        \caption{Possible Results of Prediction}
        \label{my-label}
        \begin{tabular}{lll}
        Outcome & Prediction & Result \\
        H       & H          & 1  \\
        H       & T          & 0   \\
        T       & H          & 0   \\
        T       & T          & 1    \\
        \end{tabular}
    \end{table}
    \begin{enumerate}[label=\alph*]
    \item Let $Y$ refer to the number of correct tests, and denote the outcomes of the 10 individual tests
    with the random variables $X_1,X_2, \cdots X_10$.What are the distributions of each $X_i$? What is the 
    relationship between $X_{1:10}$ and $Y$?
    For each random varibale $X_i$, it has a Bernoulli distribution to be either 1 (for correct test) or 0
    (for incorrect test). The relationship between $X_i,\ 1\leq i\leq 10$ and $Y$ is 
    \begin{equation*}
        Y = \sum_{i=1}^{10} X_i
    \end{equation*}

    \item What is the probability that he would have done at least this well if he had no ESP, i.e. if his guesses
    were essentially random?
    The event he would have done at least this well can be rewrite as $Y \geq 7$, denoting $Y$ as the number of 
    correct tests. And since $Y$ is the sum of Bernoulli random variables, it will also be a Bernoulli random variable.
    From the result table above, we can conclude that the probability to get a head in a single trial is $p = 0.5$.
    \begin{align*}
        P(Y \geq 7)     &= P(Y=7) + P(Y=8) + P(Y=9) + P(Y=10) \\
                        &= \binom{10}{7}p^7(1-p)^3 + \binom{10}{8}p^8(1-p)^2
                           + \binom{10}{9}p^9(1-p)^1 + \binom{10}{10}p^10(1-p)^0 \\
                        &\approx 0.1719
    \end{align*}

    \item Suppose the test is changed - now, the coin is flipped until the man makes an incorrect guess.
    He guesses the first two correctly, but guesses the third wrong. What is the probability of this
    experimental outcome (again, assuming no ESP)?
    Denote the event that guessing first 2 correct and the third wrong to be $E$. Then,
    \begin{equation*}
        P(E) = 0.5^2 (1-0.5)^1 = 0.125
    \end{equation*}

    \item Assume both tests were planned beforehand. What is the probability that both of these tests turned
    out the way they did? In other words, the plan was to first flip the coin 10 times and count how many times 
    the man is correct (Y ) then to continue flipping until the man makes the next mistake, at flip 10 + Z. 
    You are asked the probability that Y = 7 and Z = 3.
    \begin{align*}
        P(\{Y=7\}\cup \{Z=3\})  &= P(Y=7) \cdot P(Z=3) \\
                            &= \binom{10}{7}p^7(1-p)^3 \cdot p^2 (1-p)^1 \\
                            &\approx 0.01465
    \end{align*}
    \end{enumerate}
        %----------------------------------------------------------------------------------------
        %	APPENDIX
        %----------------------------------------------------------------------------------------

    \end{document}
    