\documentclass[]{exam}
\usepackage[utf8]{inputenc}
\usepackage{enumitem} % allows us to use the enumerate command
\usepackage{amsmath} % allows us to type in math
\usepackage{amsfonts}
\usepackage{verbatim} % allows us to type code like text
\usepackage{graphicx} % allows us to include figures
\usepackage{gensymb}
\usepackage{color}

%opening
\title{}

%===> Formatting ===>
\setlength{\parskip}{8pt}
\setlength{\parindent}{5pt}
%<=== Formatting <===


\title{Math 327 Homework 1}
\author{Chongyi Xu}

\begin{document}
	
\maketitle
\renewcommand{\labelitemii}{$\circ$}
\begin{questions}
\question Prove that the product of a nonzero rational number and an irrational number is irrational. How about the product of two irrational numbers?
\begin{parts}
	\part Let $a = \frac{m}{n},  b = \sqrt{2}, $ and $m, n \in \mathbb{N}$, then
	$a \in \mathbb{Q}, b \in \mathbb{R}\backslash\mathbb{Q}. \Rightarrow a\dot b
	= \sqrt{2}\frac{m}{n}$, which is an irrational number. Q.E.D.

	\part Let $a = \sqrt{\frac{m}{n}}, b = \sqrt{\frac{p}{q}}, m, n, p, q \in \mathbb{N}$, then 
		$a \in \mathbb{R}\backslash\mathbb{Q}, b \in \mathbb{R}\backslash\mathbb{Q}$
	\begin{align*}
	&\rightarrow if \ \sqrt{\frac{m}{n}} = \sqrt{\frac{p}{q}},  a \cdot b = \frac{m}{n} \in \mathbb{Q} \\
	&\rightarrow if \ \sqrt{\frac{m}{n}} \neq \sqrt{\frac{p}{q}}, a \cdot b = \sqrt{\frac{m\cdot p}{n\cdot q}} \in \mathbb{R} \backslash \mathbb{Q}
	\end{align*}
\end{parts}

\question  Find the inf (greatest lower bound), sup (least upper bound), max and min, if they exist, for the following sets. Prove your answers in (d) and (e). To prove m is the inf of a set, you have to show it is a lower bound and that no number bigger than m is a lower bound. To prove M is the sup of a set, you have to show it is an upper bound and that no number smaller than M is an upper bound.
\begin{enumerate}[label = (\alph*)]
	\item $S = \{1, 3, 5, 7, 9\}$
	\\ $infS = minS = 1, \ supS = maxS = 9$
	\item $S = (3, \pi]$
	\\ $infS = 3, \ supS = maxS = \pi$
	\\ min does not exist.
	\item $S =  \{q \in \mathbb{Q} : 3 < q \leq \pi\}$
	\\ $supS = \pi$
	\\ min, max, and inf do not exist.
	\item $S = \{\frac{1}{a}: a\in\mathbb{Z},\ a \neq 0\}$
	\\ $infS = minS = -1, \ supS = maxS = 1$
	\\ Proof: 
	\begin{itemize}
		\item infimum 
		\begin{itemize}
			\item $x \geq -1\ \forall\ x \in S$. Thus -1 is a lower bound.
			\item Assume -1 is not the infimum. Then $\exists r > 0,\ -1 + r = infS$
			\begin{gather*}
				0  < \frac{r}{2} < r \\
				-1  < -1 + \frac{r}{2} < -1 + r
			\end{gather*}
			\\ since $-1 + \frac{r}{2} \in S, -1 + r$ is not the lower bound. Contradicts.
			\\ $\Rightarrow$ -1 is the infimum. And Since $-1 \in S$, minS = infS
		\end{itemize}
		\item supremum
		\begin{itemize}
			\item $x \leq 1\ \forall\ x \in S$. Thus 1 is a upper bound.
			\item Assume 1 is not the supremum. Then $\exists r > 0,\ 1 - r = supS$
			\begin{gather*}
				0 < \frac{r}{2} < r \\
				0 > -\frac{r}{2} > -r \\		
				1 - r < 1 - \frac{r}{2} < 1
			\end{gather*}
			\\ since $1 - \frac{r}{2} \in S, 1 - r$ is not the upper bound. Contradicts.
			\\ $\Rightarrow$ 1 is the supremum. And Since $1 \in S$, maxS = supS
		\end{itemize}
	\end{itemize}
	\item $S = \{\frac{n+2}{2n+5}:n \in \mathbb{N}\}$
	\\$infS = minS = \frac{3}{7}, supS = \frac{1}{2}$
	\\max does not exist.
	\\Proof:
	\begin{itemize}
		\item infimum
		\begin{itemize}
			\item Since $\frac{d}{dn}\frac{n+2}{2n+5}$ is positive, the value of $\frac{n+2}{2n+5}$ will keep growing as $n$ increases. Therefore, it has its lower bound when n is smallest, in the other word, when $\ n = 1$. When$n = 1,\ x \geq \frac{3}{7}\ for \forall x \in S$. Thus $\frac{3}{7}$ is the a lower bound.
			\item Assume $\frac{3}{7}$ is not the infimum. Then $\exists r > 0,\ r + \frac{3}{7} = infS$
			\begin{gather*}
			0 < \frac{r}{2} < r \\
			\frac{3}{7} < \frac{3}{7} + \frac{r}{2} < \frac{3}{7} + r
			\end{gather*}
			\\ Since $\frac{3}{7} + \frac{r}{2} \in S,\ \frac{3}{7} + r$ is not a lower bound. Contradicts.
			\\ $\Rightarrow \frac{3}{7}$ is the infimum. And since $\frac{3}{7} \in S, minS = infS$.
		\end{itemize}
		\item supremum
		\begin{itemize}
			\item Since L'Hospital's Rule tells $\lim_{x \to +\infty}\frac{n+2}{2n+5} = \frac{1}{2}$, it has its upper bound $\frac{1}{2}$ and has no maximum.
			\item Assume $\frac{1}{2}$ is not the supremum. Then $\exists r > 0,\ \frac{1}{2} - r = supS$
			\begin{gather*}
			0 < \frac{r}{2} < r \\
			0 > -\frac{r}{2} > -r \\
			\frac{1}{2} - r < \frac{1}{2} - \frac{r}{2} < \frac{1}{2}
			\end{gather*}
			\\ Since $\frac{1}{2} - \frac{r}{2} \in S,\ \frac{1}{2} - r$ is not the uppr bound. Contradicts.
			\\ $\Rightarrow \frac{1}{2}$ is the supremum.
		\end{itemize}
	\end{itemize}
\end{enumerate}
\question Suppose A and B are non-empty sets of real numbers that are both bounded above.
\begin{enumerate}[label = (\alph*)]
	\item Prove that if $A \subset B$, then $supA \leq supB$.
	\\ Proof by contradiction: Assume $supA > supB$ (in order to reach contradiction), 
	\begin{equation*}
	\exists C = \{k \in \mathbb{R}: k \in [supB, supA]\}
	\end{equation*}
	\\ And by the definition of supremum, $C \subset A \ but\  C \not \subset B$. Contradicts the condition $A \subset B$.
	\\ Therefore, if $A \subset B$, then $supA \leq supB$. Q.E.D.

	\item Prove that $sup(A \cup B) = max\{supA, supB\}$
	\\ Proof:
	\begin{itemize}
		\item Show it is an upper bound.
		\\ Assume $supA \geq supB$, then $max\{supA, supB\} = supA$(Otherwise, switch $supA$ with $supB$)
		\\ Let $x \in A \cup B$
		\begin{align*}
		&\rightarrow if\ x \in A,\ supA \geq x \\
		&\rightarrow if\ x \in B,\ supB \geq x,\ but\ supA \geq supB,\ supA \geq x
		\end{align*}
		 Thus in any cases, $supA \geq x$. $supA$ is an upper bound.

		\item Show no smaller number works.
		\\ Let $k < supA$, then $\exists x \in A$ with $k < x \leq supA$
		\\ Since $A \subseteq A \cup B, x \in A \cup B$
		\\ So $k$ is not an upper bound for $A \cup B$
		\\ Therefore, $sup(A \cup B) = max\{supA, supB\}$ Q.E.D
	\end{itemize}

	\item Prove that if $A \cap B \neq emptyset$, then $sup(A \cap B) \leq min\{supA, supB\}$. Give an example to show that equality need not hold.
	\begin{itemize}
		\item Proof:
		\begin{itemize}
			\item Show it is an upper bound.
			\\ Assume $supA \leq supB$, then $mim\{supA, supB\} = supA$(Otherwise, switch $supA$ with $supB$)
			\\ Assume $A \cap B \neq emptyset$. Let $x \in A \cap B$. Thus $x \in A$
			\\ From the definition of supremum, $x \leq supA$. $supA$ is an upper bound. Q.E.D
		\end{itemize}

		\item example:
		\\ Two sets. $A = \{1, 3, 4, 5, 6\},\ B = \{2, 3, 5, 7\}$
		\\ Then $A \cap B = \{3, 5\}$. $supA = 6,\ supB = 7 \Rightarrow \ min(supA, supB) = 6$. But $sup(A \cap B) = 5$. The equality does not hold.
	\end{itemize}
\end{enumerate}
\end{questions}
\end{document}