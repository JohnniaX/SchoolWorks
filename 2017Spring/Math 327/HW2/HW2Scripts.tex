\documentclass[]{exam}
\usepackage[utf8]{inputenc}
\usepackage{enumitem} % allows us to use the enumerate command
\usepackage{amsmath} % allows us to type in math
\usepackage{amsfonts}
\usepackage{setspace}
\usepackage{verbatim} % allows us to type code like text
\usepackage{graphicx} % allows us to include figures
\usepackage{gensymb}
\usepackage{color}
\usepackage{commath}
\doublespacing
%opening
\title{}

%===> Formatting ===>
\setlength{\parskip}{8pt}
\setlength{\parindent}{20pt}
%<=== Formatting <===


\title{Math 327 Homework 2}
\author{Chongyi Xu}

\begin{document}
	
\maketitle
\renewcommand{\labelitemii}{$\circ$}
\begin{questions}
\question Prove the following.
	\begin{parts}
	\part For any $a, b$ real numbers, $|a| - |b| \leq |a + b|$.
	\\ Let $b = b + a - a$, then $|b| = |b + a - a| \leq |a + b| - |a|$(By Triangular Inequality)
	\\ Since $-|b| \leq +|b|$, so $-|b| \leq |b| \leq |a + b| - |a|$
	\\ Thus $|a| - |b| \leq |a| + |b| \leq |a + b|$ Q.E.D.\\

	\part For any $a, b$ real numbers, $||a| - |b|| \leq |a + b|$.
	\begin{align*}
	&\rightarrow \text{if}\ |a| - |b| \geq 0,\ \text{part(a) tells it is true}.\\
	&\rightarrow \text{if}\ |a| - |b| < 0,\ ||a| - |b|| = |b| - |a| \leq |b + a|\text{(By part(a))}.
	\end{align*}
	In both cases $||a| - |b|| \leq |a + b|$ Q.E.D.\\

	\part For any $a, b$ real numbers, $||a| - |b|| \leq |a - b|$.
	\\ Let $b = -c,\ c \in \mathbb{R}$. In part(b), it is proved that $||a| - |c|| \leq |a + c|$. So in this case, let $c = -b$, then we have $||a| - |b|| \leq |a - b|$ since $|-c| = c$. Q.E.D
	\end{parts}

\question Prove Bernoulli's Inequality
	\begin{equation*}
		(1 + b) ^ n \geq 1 + nb
	\end{equation*}
	in two different ways:
	\begin{parts}
	\part For any $b \geq 0$, using the binomial formula.
		\\ Binomial Formula tells that $(a + b) ^ n = \sum_{k = 0}^{n} {{n}\choose{k}} a^{n - k}b ^ k$. So we have
		\\ Let $a = 1$, 
		\begin{align*}
		& (1 + b) ^ n = \sum_{k = 0}^{n} {{n}\choose{k}} b ^ k\\
		&\rightarrow \text{when}\ k = 0,\ {{n}\choose{0}}\cdot b ^ 0 = 1\\
		&\rightarrow \text{when}\ k = 1,\ {{n}\choose{1}}\cdot b ^ 1 = nb
		\end{align*}
		So $(1 + b) ^ n = 1 + nb + \sum_{k = 2}^{n} {{n}\choose{k}}b ^ k$.
		\\ Since given $b \geq 0$, then $\sum_{k = 2}^{n} {{n}\choose{k}}b ^ k \geq 0$.
		\\ Therefore $(1 + b) ^ n \geq 1 + nb$. Q.E.D.

	\part For any $b > -1$, using mathematical induction.
		\\ Assume $b > -1$. 
		\begin{itemize}
			\item Base Case($n = 0$)\\
			$(1 + b) ^ 0 = 1 \geq 1 + 0 \cdot b$
			\item Inductive Step
			\\ Assume $(1 + b) ^ n \geq 1 + nb$
			\\ Prove $(1 + b) ^ {n + 1} \geq 1 + (n + 1) \cdot b$
			\\ $\bullet$ Proof: 
			\begin{align*}
			(1 + b) ^ {n + 1} &= (1 + b) ^ {n} \cdot (1 + b)\\
			\text{Since}\ b &> -1\ \text{by assumption}\\
			(1 + b) ^ {n} \cdot (1 + b) &\geq (1 + nb)(1 + b) \text{(inductive hypothesis)}\\
										  &\geq 1 + b + nb + nb^2\\
										  &\geq 1 + (n + 1)b + nb^2
			\end{align*}
			\\ Since $b ^ 2 \geq 0,\ nb ^ 2 \geq 0$
			\\ Therefore $(1 + b) ^ {n + 1} \geq 1 + (n + 1)b$. Q.E.D.
		\end{itemize}
	\end{parts}

\question Decide if the following are true or false. If true, give a short proof. If false, find a counter example.
	\begin{parts}
	\part If the sequence $|a_n|$ converges, then so does $(a_n)$
	\\ \textit{True}. Assume $|a_n|$ converges, then $\exists N \in \mathbb{N}\ \text{such that}$
	\\ $||a_n| - a| < \varepsilon$ when $n \geq N$ for $\varepsilon > 0$
	\\
	\\ $\rightarrow$ if $a_n \geq 0$, then $|a_n - a| < \varepsilon$
	\\ Converges to $a \surd$
	\\
	\\ $\rightarrow$ if $a_b < 0$
	\begin{align*}
			|-a_n - a|  &< \varepsilon\\
			|a_n + a|  &< \varepsilon\\
			|a_n - (-a)|&< \varepsilon\\
			\text{Let } b = -a,\\
			|a_n - b|   &< \varepsilon&&
	\end{align*}
	Converges to b. Q.E.D.
	\\ $\Rightarrow$ Therefore, $(a_n)$ also converges if $|a_n|$ converges.

	\part If the sequence $(a_n + b_n)$ converges, then so do the sequences $(a_n)$ and $(b_n)$.
	\\ \textit{False}. Let $a_n = 2n,\ b_n = -2n$, then $(a_n + b_n)$ converges to 0, but$a_n,\ b_n$ do not.

	\part If the sequences $(a_n + b_n)$ and $(a_n)$ converge, then so does the sequence $(b_n)$.
	\\ \textit{True}. Assume $|a_n + b_n - c| < \frac{\varepsilon}{2},\ |a_n - a| < \frac{\varepsilon}{2}$ for any $n \geq N$, where $N \in \mathbb{N}$
	\\ Let $c = a + b$, then we have $|a_n + b_n - (a + b)| < \frac{\varepsilon}{2}$
	\begin{equation*}
	\begin{split}
	|b_n - b| &= |a_n - a -a_n + a + b_n -b|\\
			  &= |a_n - a + b_n - b + (a - a_n)|\\
			  &\leq |a_n - a + b_n - b| + |a - a_n| \text{(Triangular Inequality)}\\
	\text{Since } |a - a_n| = |a_n - a|\\
	|b_n - b| &\leq |a_n - a + b_n - b| + |a_n - a|\\
			  &< \frac{\varepsilon}{2} + \frac{\varepsilon}{2}\\
			  &< \varepsilon
	\end{split}
	\end{equation*}
	\\ Therefore, $(b_n)$ converges. Q.E.D.
	\end{parts}

\question Use the definition of convergence to show the following limits.
	\begin{parts}
	\part $\lim_{n \to \infty} \frac{1}{\sqrt{n}} = 0$
	\\ $\bullet$ Proof: Given $\varepsilon > 0$, Archimedean Property(2) tells that there exists a $M \in \mathbb{N},\ \frac{1}{M} < \varepsilon$. Let $N = M ^ 2$, then $\frac{1}{\sqrt{N}} < \varepsilon$
	\\ Assume $n \geq N$, then
	\begin{equation*}
	\begin{split}
	\abs{\frac{1}{\sqrt{n}} - 0} &= \abs{\frac{1}{\sqrt{n}}}\\
							 &\leq \frac{1}{\sqrt{n}}\\
							 &\leq \frac{1}{\sqrt{N}} \text{, since }n\geq N\\
							 &< \varepsilon
	\end{split}
	\end{equation*}
	\\Therefore, $\frac{1}{\sqrt{n}}$ converges to 0 $\iff \lim_{n \to \infty} \frac{1}{\sqrt{n}} = 0$.

	\part $\lim_{n \to \infty} \frac{n ^ 2}{n ^ 2 + n} = 1$
	\\ Prove $|\frac{n ^ 2}{n ^ 2 + n} - 1| < \varepsilon$, for $n \geq N$
	\\ $\bullet$ Proof: Given $\varepsilon > 0$, Archimedean Property(2) tells that there exists a $N \in \mathbb{N},\ \frac{1}{N} < \varepsilon$. And Archimedean Property(1) tells that there exists a $N + 1 \in \mathbb{N}$. So $\frac{1}{N + 1} < \frac{1}{N} < \varepsilon$.
	\\ Assume $n \geq N \iff n + 1 \geq N + 1$, then
	\begin{equation*}
	\begin{split}
	\abs{\frac{n ^ 2}{n ^ 2 + n} - 1} &= \abs{\frac{n ^ 2 - n ^ 2 - n}{n ^ 2 + n}}\\
								  &= \abs{\frac{-n}{n ^ 2 + n}}\\
								  &= \abs{\frac{-1}{n + 1}}\\
								  &= \abs{\frac{1}{n + 1}}\\
								  &\leq \frac{1}{n + 1}\\
								  &\leq \frac{1}{N + 1}\\
								  &< \frac{1}{N}\\
								  &< \varepsilon
	\end{split}
	\end{equation*}
	\\ Therefore, $\frac{n ^ 2}{n ^ 2 + n}$ converges to 1 $\iff \lim_{n \to \infty} \frac{n ^ 2}{n ^ 2 + n} = 1$
	\end{parts}

\question Discuss the convergence of the sequence $(\sqrt{n + 1} - \sqrt{n})_{n\in\mathbb{N}}$
	\\ \textit{Claim} $(\sqrt{n + 1} - \sqrt{n})_{n\in\mathbb{N}}$ converges.
	\\ $\bullet$ Proof: Given $\varepsilon > 0$, Archimedean Property(2) tells that there exists a $M \in \mathbb{N},\ \frac{1}{M} < 2\varepsilon$. Let $M = \sqrt{N}$, then $\frac{1}{\sqrt{N}} < 2\varepsilon \Rightarrow \frac{1}{2\sqrt{N}} < \varepsilon$.
	\\ Assume $n \geq N$, then
	\begin{equation*}
	\begin{split}
	\abs{\sqrt{n + 1} - \sqrt{n}} &= \abs{\frac{n + 1 - n}{\sqrt{n + 1} + \sqrt{n}}}\\
									  &= \abs{\frac{1}{\sqrt{n + 1} + \sqrt{n}}}\\
									  &< \frac{1}{2\sqrt{n}}\\
									  &\leq \frac{1}{2\sqrt{N}}\\
									  &< \varepsilon
	\end{split}
	\end{equation*}
	\\ Therefore, $(\sqrt{n + 1} - \sqrt{n})_{n\in\mathbb{N}}$ converges to 0. 

\question Let $a_1 = 1$ and for $n \geq 1$, 
	\begin{equation*}
		a_{n + 1} =  
			\begin{cases}
			a_n + \frac{1}{n}  & \text{if } a_n ^ 2 \leq 2\\
			a_n - \frac{1}{n}  & \text{if } a_n ^ 2 > 2
			\end{cases}
	\end{equation*}
	Show that for every $n$, $\abs{a_n - \sqrt{2}} < \frac{2}{n}$ and prove that the sequence converges to $\sqrt{2}$. 
	\\ Since $a_1 = 1$ and $n \geq 1 \iff \frac{1}{n} \leq 1$, $a_n > 0$
	

\question For a sequence $(a_n)$ of positive numbers, prove that 
	\begin{equation*}
		a_n \to \infty\ \text{if and only if } \frac{1}{a_n} \to 0.
	\end{equation*}
	\begin{itemize}
		\item ($\Rightarrow$) Assume $a_n \to \infty$, prove $\frac{1}{a_n} \to 0$.
		\\ Proof: $a_n \to \infty$ implies that for every $\varepsilon > 0$, $\abs{a_n - 0} > \varepsilon$ when $n \geq N$.
		\\ So we have 
		\begin{align*}
			\abs{a_n} &> \varepsilon\\
			a_n &> \varepsilon \text{, since $a_n$ are all positive numbers}\\
			\frac{1}{a_n} &< \frac{1}{\varepsilon}
		\end{align*}
		Therefore, $\frac{1}{a_n} \to 0$ since for every $\frac{1}{\varepsilon} > 0$, $\abs{\frac{1}{a_n}} < \frac{1}{\varepsilon}$ when $n \geq N$.
		\item ($\Leftarrow$) Assume $\frac{1}{a_n} \to 0$, prove $a_n \to \infty$.
		\\ Proof: $\frac{1}{a_n} \to 0$ implies for every $\varepsilon > 0$, there exists $N\in \mathbb{N}$ such that $\abs{\frac{1}{a_n} - 0} < \varepsilon$ when $n \geq N$. 
		\\ So we have
		\begin{align*}
			\abs{\frac{1}{a_n}} &< \varepsilon\\
			\frac{1}{a_n} &< \varepsilon \text{, since $a_n$ are all positive numbers}\\
			a_n &> \frac{1}{\varepsilon}
		\end{align*}
		Therefore, $a_n \to \infty$ since $a_n$ will always be greater than $\frac{1}{\varepsilon}$, where $\varepsilon > 0$.
	\end{itemize}
	Q.E.D.
\end{questions}
\end{document} 